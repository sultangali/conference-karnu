% =====================================================================
% АВТОРЛАРҒА МАҢЫЗДЫ:
% Осы шаблонда қарастырылған командалар мен ортақтарды
% ТЕК қана пайдаланыңыз.
% Қосымша пакеттерді қоспаңыз және командаларды қайта анықтамаңыз.
%
% Атап айтқанда:
% - \author, \affiliation, \BuketovTitle, \Buketovmaketitle командаларын қолданыңыз
% - қажет болған жағдайда \begin{definition}, \begin{theorem} және т.б. пайдаланыңыз
% - әдебиеттерге сілтеме жасау үшін \cite{} және \bibitem{} қолданыңыз
%
% Қосымша пакеттер немесе макростар жинақты құрастыру кезінде
% компиляция қателеріне әкелуі мүмкін.
%
% Тезистің ең жоғарғы көлемі — 1 баспа бет.
% =====================================================================



\documentclass[a4paper,11pt]{article}
\usepackage{buketov_mit}

\begin{document}
\selectkazakh

\author{Тегі}{А.Ә.}{1}
\author{Тегі}{А.Ә.}{2}

\BuketovTitle{Атауы}

\affiliation{Ұйым атауы, Қала, Ел}{1}
\affiliation{Ұйым атауы, Қала, Ел}{2}

\mail{email@domain.com}

\Buketovmaketitle


\begin{definition}
Анықтаманың мәтіні.
\end{definition}

\KeyWords{3--10 кілт сөз.}

\Funding{Бұл зерттеу ... қаржыландырылды (грант AP00000000).}

\begin{thebibliography}{9}
% Кітап
\bibitem{cite1} Тегі, А.~Ә. (жыл). \textit{Дереккөз атауы}. Баспасы.

% Мақала
\bibitem{cite2} Тегі, А.~Ә. (жыл). Дереккөз атауы.
\textit{Журнал}, \textit{Том}(Нөмір), 123--145.
\end{thebibliography}

\end{document}
