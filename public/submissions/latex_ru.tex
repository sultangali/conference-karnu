% =====================================================================
% ВАЖНО ДЛЯ АВТОРОВ:
% Пожалуйста, используйте ТОЛЬКО команды и окружения,
% предусмотренные данным шаблоном.
% Не подключайте дополнительные пакеты и не переопределяйте команды.
%
% В частности:
% - используйте \author, \affiliation, \BuketovTitle, \Buketovmaketitle
% - при необходимости используйте \begin{definition}, \begin{theorem} и др.
% - для ссылок используйте \cite{} и \bibitem{}
%
% Добавление собственных пакетов или макросов может привести
% к ошибкам компиляции при формировании сборника.
%
% Максимальный объём тезиса -- 1 печатная страница.
% =====================================================================



\documentclass[a4paper,11pt]{article}
\usepackage{buketov_mit}

\begin{document}
\selectlanguage{russian}

\author{Фамилия}{И.О.}{1}
\author{Фамилия}{И.О.}{2}

\BuketovTitle{Название}

\affiliation{Организация, Город, Страна}{1}
\affiliation{Организация, Город, Страна}{2}

\mail{email@domain.com}

\Buketovmaketitle


\begin{definition}
Текст определения
\end{definition}

\KeyWords{3--10 ключевых слов.}

\Funding{Данное исследование было профинансировано ... (грант AP00000000).}

\begin{thebibliography}{9}
%Книга
\bibitem{cite1} Фамилия, И.~О. (год). \textit{Название источника}. Издательство.

%Статья
\bibitem{cite2} Фамилия, И.О. (год). Название источника. \textit{Журнал}, \textit{Том}(Номер), 123--145.

\end{thebibliography}

\end{document}
